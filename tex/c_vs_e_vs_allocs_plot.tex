In Fig. \ref{fig:compare_delta_E} we show how the design space exploration can be sped up by skipping some resource allocations solely based on the resulting minimum II.
We define

\begin{equation}
    \label{eq:tradeoff_alloc}
    C = \frac{M}{speedup} = \frac{M \cdot \ratII}{\intII} = \frac{M^2}{S \cdot \ceil{\frac{M}{S}}} 
\end{equation}

with

\begin{equation}
    \label{eq:speedup}
    speedup = \frac{\intII}{\ratII}
\end{equation}

as a cost metric to decide whether to start the HLS flow or skip the allocation altogether. Large values for $M$ produce high interconnect costs due to large MUXs in the data path. 
Therefore, we choose to skip those allocations if the speedup due to choosing a rational II does not justify the MUX cost increase.

\begin{figure}[htbp]
    \centering
\begin{tikzpicture}
\pgfplotsset{
    %scale only axis,
    %scaled x ticks=base 10:3,
    xmin=2, xmax=12,
    xtick={2,3,4,5,6,7,8,9,10,11,12}
}

\begin{axis}[
  axis y line*=left,
  ymin=0, ymax=35,
  xlabel=max. $C$,
  ylabel={$\Delta E_s$ [\%]},
  ytick={0,5,10,15,20,25,30,35},
  width = 0.9\linewidth,
  height= 90mm,
]
\addplot[
    thick,
    red,
    %c0,
    solid,
    %only marks, 
    mark=square, 
    mark options={solid, scale=1.0},
    ] 
    table [x=maxquot, y=avg,col sep = semicolon] {../synth_results/C_vs_E_vs_allocs/delta_E_all.csv}; \label{plot_one}
%\addlegendimage{/pgfplots/refstyle=Hplot}
%\addlegendentry{plot 1}

\addplot[
    thick,
    %green,
    c2,
    dashed,
    %only marks, 
    mark=triangle, 
    mark options={solid, scale=1.5},
    ] 
    table [x=maxquot, y=max,col sep = semicolon] {../synth_results/C_vs_E_vs_allocs/delta_E_all.csv}; \label{plot_two}
%\addlegendentry{plot 2}
\end{axis}

\begin{axis}[
  axis y line*=right,
  axis x line=none,
  ymin=0, ymax=200,
  ylabel=\# allocs,
  ytick={0,25,50,75,100,125,150,175,200},
  width = 0.9\linewidth,
  height= 90mm,
  legend style={at={(0.575,0.05)},anchor=south west},
  legend cell align = left,
]
\addlegendimage{/pgfplots/refstyle=plot_one}
\addlegendentry{avg. $\Delta E_s$}
\addlegendimage{/pgfplots/refstyle=plot_two}
\addlegendentry{max. $\Delta E_s$}
\addplot[
    thick,
    dotted,
    blue,
    %c2,
    %only marks, 
    mark=x, 
    mark options={solid, scale=1.5},
    %mark options={scale=1.8, solid, draw=blue}
    ] 
    table [x=maxquot, y=numAllocs,col sep = semicolon] {../synth_results/C_vs_E_vs_allocs/delta_E_all.csv}; \label{plot_three} 
\addlegendentry{\# allocs}
\end{axis}

\end{tikzpicture}
    \caption{Comparing the average and maximum energy savings for various upper bounds of $C$. All allocations with $C > \mathrm{max. }\; C$ are skipped.}
    \label{fig:compare_delta_E}
\end{figure}

We see that increasing $C$ beyond 5 is pointless, since the maximum energy savings do not increase anymore but the average energy savings decrease and the number of considered allocations increases. 
